\section{系统需求分析}

\subsection{系统目的和意义}

\subsubsection{系统目的:}
进行毕业设计需求分析、完善毕设管理流程。
在此基础上进行整体技术架构设计、功能设计和数据库设计,
并对系统进行实现。
旨在构建一个高效、易用的毕业设计管理系统,为数理学院教务工作提供便利。

\subsubsection{系统意义:}
计算机技术与网络技术日新月异的发展,
推动了信息技术在社会各个领域的广泛应用,
也为教育带来了不可估量的影响,
而且人们的上网时空已经不受限制,
随着移动平台的普及,
手机、iPad等产品越来越多地被使用。
尤其在教育领域,目前,我国许多高校的各项管理工作都在向信息化建设迈进,
以校园网为平台的“数字化”校园建设情况已经成为衡量高校信息化建设水平的一个标志。
而作为高校重点管理工作的毕业设计管理,其管理水平是否先进对学校办学质量具有举足轻重的影响。
尤其在校园网络的基础上,
建立高校毕业设计管理模式和信息化管理平台,
开发具有自身院校特点的毕业设计管理系统非常重要。


\subsection{功能模块划分}

\subsubsection{登录窗口}
在此毕业设计管理系统中,
我们设计了管理员端、教师端以及学生端三个端,
通过在登陆界面输入不同的账户名可以分别登入各个端。

\subsubsection{管理员端}
在管理员端中,我们预计开设五个模块,分别为:
参数设置、课题审核、选题操作、选题结果以及修改密码模块。
分别用于实现对专业、教师提交题目截止时间、管理员审核题目截止时间、
学生第一次选题开始时间、学生第一次选题截止时间、管理员第一次匹配截止时间
学生第二次选题截止时间、管理员第二次匹配截止时间进行设置;
查看课题的编号、名称、指导老师、所属专业年级以及审核情况,
对课题进行审核操作;
设置提前批选题的学号以及题号,
查看第一第二次选题分配情况,
导出选题匹配结果以及选题失败同学名单;
查看学生选题的课题编号、学号、姓名、课题名称、指导老师以及学年的信息,
按年份筛选、搜索课题和导出选题结果表格;
重置教师和学生的密码的功能。

\subsubsection{教师端}
在教师端中,拟开设四个模块,分别为:
课题提交、选题列表、选题结果和修改密码模块。
分别用于输入课题的各项基本信息;
查看提交的课题的信息并进行修改删除等操作;
查看学生的选题结果信息,包括学生姓名学号专业班级联系方式;
最后提供了教师更新密码的窗口。

\subsubsection{学生端}
在管理员端中,我们预计开设五个模块,分别为:
选题规则、选题操作、预选志愿、选题结果以及修改密码模块。
学生登入后就会自动跳转到选题规则页面,让学生们了解选题的具体规则,
减少出现因不熟悉规则而导致选题出现的错误;
选题操作界面可以查看课题的内容信息,仅包含课题的名称、编号以及类别信息,
防止学生们选老师而不是选课题;
在预选志愿界面,学生可以查看自己预选的第一到第四志愿,以及对应课题的信息,
方便学生进行检查确认;
在选题结果模块中,学生可以查看自己最后选择到的课题的课题编号、名称信息,
并且可以查看自己课题的指导老师;
修改密码界面提供了平台供同学们更新自己的密码。


\subsection{系统开发环境}

\begin{itemize}
    \item 前端:
          \begin{itemize}
              \item node.js
              \item React
              \item Redux
              \item JavaScript
          \end{itemize}
    \item 后端:
          \begin{itemize}
              \item Python
              \item FastAPI
              \item SQLAlchemy
              \item Alembic
          \end{itemize}
\end{itemize}


\section{系统概念结构设计}

\subsection{登录界面}
\begin{itemize}
    \item 管理员账户
    \item 教师账户
    \item 学生账户
\end{itemize}

\subsection{管理员端}
\begin{itemize}
    \item 参数设置
    \item 课题审核
    \item 选课操作
    \item 选课结果
    \item 修改密码
\end{itemize}

\subsection{教师端}
\begin{itemize}
    \item 课题提交
    \item 选课列表
    \item 选课结果
    \item 修改密码
\end{itemize}

\subsection{学生端}
\begin{itemize}
    \item 选课规则
    \item 选课操作
    \item 预选志愿
    \item 选课结果
    \item 修改密码
\end{itemize}
